
\chapter{Conclusiones}

\label{ch:conclusions}

\section{La matemática aplicada}

Las asignaturas que más disfruté en la carrera eran las de matemática aplicada. Y el motivo de ello se puede ver reflejado en este trabajo. 

Estamos ante una rama que une la matemática con nuestra realidad, con aplicaciones de teorías que dan respuestas a preguntas de nuestro alrededor. Pero, lo que es más fascinante, recurre de todas las demás ramas de la matemática, a multitud de herramientas.

Durante este proyecto he trabajado con análisis real, complejo, numérico, geometría, topología, computación...
Y aunque esto haya supuesto una tarea por la que me haya podido sentir sobrepasado en muchos casos, ha sido genial poder ver en armonía y aplicación tantos conceptos.

Además, me gustaría resaltar como sólo he vislumbrado un bosquejo del gran campo de las bifurcaciones, o de la teoría de continuación, o incluso de partes tan particulares como las funciones test. Sin duda uno de las mayores dificultades a las que me enfrenté fue la dificultad de idear un trabajo que sintetizase todo lo que creía conveniente contar, aportando ejemplos reales y algunas demostraciones para mostrar una buena comprensión. 

Al final del mismo, puede que mi enfoque no sea el más correcto, o que existan, según la persona que esté leyendo esto, otras formas de contar y presentar lo expuesto. Aun así, estoy satisfecho con el resultado y con la cantidad de conceptos aprendidos. 



\section{El software matemático}
Una de las primeras conclusiones a destacar es la necesaria renovación del software matemático.
Es habitual que en clase se nos enseñe con programas de código libre, pero esto presenta dos inconvenientes:
Primero, exige estar en constante cambio y evolución. Esto se puede notar en los códigos de este trabajo, alejados del software que nos enseñan, en pos de lenguajes que evolucionan y cada vez tienen más y mejores paquetes. 
Segundo, es habitual que los programas de software libre especializados no contengan tanto soporte y documentación como los privativos.

Lejos de ser una crítica, me parecen puntos a mejorar. AUTO es un lenguaje con una curva de aprendizaje que al principio parece tener pendiente infinita cuya última actualización fue en 2007. Y aún así, la mayoría de aplicaciones son intérpretes del mismo.

Surge como alternativa XPPAUT, y aunque es algo más agradable en cuanto a la creación de código, visualmente deja mucho que desear.


\section{El arte de encontrar parámetros}
Durante este proyecto me he enfrentado a multitud de escollos que he tenido que ir superando, pero, sin ninguna duda, las mayores decepciones han venido de la mano de los parámetros.

Uno puede pensar que la realización de diagramas de bifurcaciones de modelos sencillos no debiera er complicada, pero se equivoca. Es habitual que en la literatura no se trabajen con estos diagramas, ya sea por la dificultad y horas que debemos emplear en ellos o, por supuesto, por claridad, puesto que habitualmente es necesario tener un bagage matemático extenso para que se entienda que estamos representando. 

Nos encontramos entonces ante un camino árduo y muchas veces desechado.

La otra opción podría consistir en consultar papers publicados. Sin embargo, aunque en ellos suelan venir los datos de los parámetros (recalco \emph{los parámetros}, pues para escribir el programa son necesarios muchos más datos como la longitud de paso o las condiciones iniciales), en general los sistemas que hoy en día se tratan poseen varios parámetros y muchas mas ecuaciones.

Este trabajo me ha dado una perspectiva nueva sobre el análisis numérico, mostrándome como la matemática aplicada debe nutrirse de esfuerzo, ensayos, errores y tiempo para que podamos obtener resultados. También ha supuesto un refrescante contacto con la vida real, en la que a menudo no se buscan soluciones exactas (porque no suelen encontrarse) y en la que las respuestas (que ya no son blanco o negro) deben estar sometidas a constante revisión.



